\documentclass[11pt, a4paper]{article}


\usepackage{soul}
\usepackage{natbib}
\usepackage{hyperref}
\usepackage{graphicx}             
\graphicspath{{./Figuras/}}

\usepackage{makecell}
\usepackage[margin=1.0in]{geometry}
\usepackage{float}                
\usepackage{amsmath}
\usepackage{amscd}
\usepackage{amsfonts}
\usepackage{amssymb}
\usepackage{bbm}
\usepackage{booktabs}
\usepackage{nameref}
\usepackage{multirow}
\usepackage[nokeyprefix]{refstyle}
\usepackage{rotating}
\usepackage{threeparttable}
\usepackage{lscape}
\usepackage{enumerate}
\usepackage{afterpage}
\usepackage{caption}
\usepackage{subcaption}
\usepackage{epstopdf}
\epstopdfDeclareGraphicsRule{.tiff}{png}{.png}{convert #1 \OutputFile}
\AppendGraphicsExtensions{.tiff}

\epstopdfDeclareGraphicsRule{.tif}{png}{.png}{convert #1 \OutputFile}
\AppendGraphicsExtensions{.tif}

\usepackage{tikz}
\usetikzlibrary{shapes.geometric, arrows}
\usetikzlibrary{calc}
\usetikzlibrary{matrix}

\tikzset{ 
    table/.style={
        matrix of nodes,
        row sep=-\pgflinewidth,
        column sep=-\pgflinewidth,
        nodes={
            rectangle,
            draw=black,
            align=center
        },
        minimum height=1.5em,
        text depth=0.5ex,
        text height=2ex,
        nodes in empty cells,
%%
        every even row/.style={
            nodes={fill=gray!20}
        },
        column 1/.style={
            nodes={text width=2em,font=\bfseries}
        },
        row 1/.style={
            nodes={
                fill=black,
                text=white,
                font=\bfseries
            }
        }
    }
}


\usepackage{colortbl}

\newtheorem{theorem}{Theorem}
\newtheorem{claim}[theorem]{Claim}



%%% HELPER CODE FOR DEALING WITH EXTERNAL REFERENCES
\usepackage{xr}
\makeatletter
\newcommand*{\addFileDependency}[1]{
  \typeout{(#1)}
  \@addtofilelist{#1}
  \IfFileExists{#1}{}{\typeout{No file #1.}}
}
\makeatother


\newcommand*{\myexternaldocument}[1]{
    \externaldocument{#1}
    \addFileDependency{#1.tex}
    \addFileDependency{#1.aux}
}

\newgeometry{margin=1in}
\linespread{1.25}
\begin{document}


\title{The controlled choice design and private paternalism in pawnshop borrowing - STATA documentation}
\author{Craig McIntosh \and Isaac Meza \and Joyce Sadka \and Enrique Seira \and Francis J.\ DiTraglia   \thanks{Seira:  MSU, \url{enrique.seira@gmail.com} (corresponding author); McIntosh:  University of California San Diego, \url{ctmcintosh@ucsd.edu}; Meza: Harvard University, \url{isaacmezalopez@g.harvard.edu}; Sadka: ITAM, \url{jsadka@itam.mx}; DiTraglia: Oxford, \url{francis.ditraglia@economics.ox.ac.uk}} }
\date{}
\maketitle


\section{ Derivations for Section \ref{sec:randchoice}}
\label{append:randchoice}


This appendix provides proofs of the results described in Section \ref{sec:randchoice}, using the notation and assumptions described in Section \ref{sec:potentialOutcomes}. 
To simplify the presentation, we omit $i$ subscripts throughout this section.
We also use the shorthand $Z_0 \equiv \mathbbm{1}(Z=0)$, $Z_1 \equiv \mathbbm{1}(Z=1)$, and $Z_2 \equiv\mathbbm{1}(Z=2)$.
For convenience, the following assumption collects our exclusion restriction and the key features of the constrained choice design.

\begin{assumption}\mbox{}
\label{assump:randchoice}
   \begin{enumerate}[(i)]
   \item $Z$ is independent of $(Y_{0}, Y_{1}, C)$
   \item $D = \mathbbm{1}(Z \neq 2)Z + \mathbbm{1}(Z = 2)C$
   \item $Y = \mathbbm{1}(Z = 0) Y_{0} + \mathbbm{1}(Z = 1) Y_{1} + \mathbbm{1}(Z = 2) [ (1 - C) Y_{0} + C Y_{1}]$
   \end{enumerate}
\end{assumption}


\subsection{Point Identification}

We first show that the TOT, TUT, ASB, and ASL effects are point identified under the constrained choice design.
It follows that the ASG effect, $(\text{TOT} - \text{TUT})$, is likewise point identified. 

\begin{lem}
Under Assumption \ref{assump:randchoice},
\label{lem_randchoice}
   \begin{enumerate}[(i)]
       \item $\mathbbm{E}(D|Z=2) = \mathbb{P}(C=1)$
       \item $\mathbbm{E}(Y|Z=0) = \mathbbm{E}(Y_0)$
       \item $\mathbbm{E}(Y|Z=1) = \mathbbm{E}(Y_1)$
       \item $\mathbbm{E}(Y|D=0,Z=2) = \mathbbm{E}(Y_0|C=0)$
       \item $\mathbbm{E}(Y|D=1,Z=2) = \mathbbm{E}(Y_1|C=1)$.
   \end{enumerate} 
\end{lem}

\begin{proof}
Part (i) follows because $Z=2$ implies $D=C$ and $Z$ is independent of $C$.  
Parts (ii) and (iii) follow similarly: given $Z=0$ we have $Y = Y_0$, given $Z=1$ we have $Y = Y_1$, and $Z$ is independent of $(Y_0,Y_1)$.
For parts (iv) and (v), first note that Assumption \ref{assump:randchoice} (iii) implies that $Z$ is conditionally independent of $(Y_0,Y_1)$ given $C$.
Now, $Z=2$ implies that $D=0$ if and only if $C=0$. Hence,
\[
\mathbbm{E}(Y|D=0, Z=2) = \mathbbm{E}(Y_0|D=0, Z=2) = \mathbbm{E}(Y_0|C=0, Z=2)=\mathbbm{E}(Y_0|C=0)
\]
establishing part (iv).
For part (v) $Z=2$ implies that $D=1$ if and only if $C=1$ and hence
\[
\mathbbm{E}(Y|D=1, Z=2) = \mathbbm{E}(Y_1|D=1, Z=2) = \mathbbm{E}(Y_1|C=1, Z=2)= \mathbbm{E}(Y_1|C=1). %\qedhere
\]
\end{proof}

\begin{prop} 
Under Assumption \ref{assump:randchoice},
    \begin{enumerate}[(i)]
        \item $\text{TOT} \equiv \mathbbm{E}(Y_1 - Y_0|C=1)  = \displaystyle \frac{\mathbbm{E}(Y|Z=2) - \mathbbm{E}(Y|Z=0)}{\mathbbm{E}(D|Z=2)}$
        \item $\text{TUT} \equiv \mathbbm{E}(Y_1 - Y_0|C=0) = \displaystyle \frac{\mathbbm{E}(Y|Z=1) - \mathbbm{E}(Y|Z=2)}{1 - \mathbbm{E}(D|Z=2)}$
        \item $\text{ASB} \equiv \mathbbm{E}(Y_0|C=1) - \mathbbm{E}(Y_0|C=0) = \displaystyle \frac{\mathbbm{E}(Y|Z=0) - \mathbbm{E}(Y|Z=2,D=0)}{\mathbbm{E}(D|Z=2)}$
        \item $\text{ASL} \equiv \mathbbm{E}(Y_1|C=1) - \mathbbm{E}(Y_1|C=0) = \displaystyle \frac{\mathbbm{E}(Y|Z=2,D=1) - \mathbbm{E}(Y|Z=1)}{1 - \mathbbm{E}(D|Z=2)}$.
    \end{enumerate}
\end{prop}

\begin{proof}
For parts (i) and (iii) we require an expression for $\mathbbm{E}(Y_0|C=1)$ in terms of the observables $(Y, D, Z)$.
By Lemma \ref{lem_randchoice}(ii) and iterated expectations 
\[
\mathbbm{E}(Y|Z=0) = \mathbbm{E}(Y_0) = \mathbbm{E}(Y_0|C=0) \mathbbm{P}(C=0) + \mathbbm{E}(Y_0|C=1) \mathbbm{P}(C=1).
\]
Re-arranging and substituting Lemma \ref{lem_randchoice}(i) and (iv),
\begin{align}
\mathbbm{E}(Y_0|C=1)  &= \frac{\mathbbm{E}(Y|Z=0) - \mathbbm{E}(Y_0|C=0) \mathbbm{P}(C=0)}{\mathbbm{P}(C=1)}\nonumber\\ 
&=  \frac{\mathbbm{E}(Y|Z=0) - \mathbbm{E}(Y|Z=2,D=0) \mathbbm{E}(1 - D|Z=2)}{\mathbbm{E}(D|Z=2)}.
\label{eq:Y0C1}
\end{align}
Part (i) follows by combining \eqref{eq:Y0C1} with Lemma \ref{lem_randchoice}(v) and simplifying; part (iii) follows by combining \eqref{eq:Y0C1} with Lemma \ref{lem_randchoice}(iv) and simplifying.
Similarly, for parts (ii) and (iv) we require an expression for $\mathbbm{E}(Y_1|C=0)$ in terms of observables.
By Lemma \ref{lem_randchoice}(iii) and iterated expectations,
\[
\mathbbm{E}(Y|Z=1) = \mathbbm{E}(Y_1) = \mathbbm{E}(Y_1|C=0)\mathbbm{P}(C=0) + \mathbbm{E}(Y_1|C=1) \mathbbm{P}(C=1).
\]
Re-arranging and substituting Lemma \ref{lem_randchoice}(i) and (v),
\begin{align}
\mathbbm{E}(Y_1|C=0) 
&= \frac{\mathbbm{E}(Y|Z=1) - \mathbbm{E}(Y_1|C=1)\mathbbm{P}(C=1)}{\mathbb{P}(D=0)} \nonumber\\
&=\frac{\mathbbm{E}(Y|Z=1) - \mathbbm{E}(Y|Z=2,D=1)\mathbbm{E}(D|Z=2)}{\mathbb{E}(1-D|Z=2)}.
\label{eq:Y1C0}
\end{align}
Part (ii) follows by combining \eqref{eq:Y1C0} with Lemma \ref{lem_randchoice}(iv) and simplifying; part (iv) follows by combining \eqref{eq:Y1C0} with Lemma \ref{lem_randchoice}(v) and simplifying.
\end{proof}

\subsection{Regression-based Estimation of TOT, TUT, ASG, ASL, and ASB}

We now show how a collection of just-identified, linear IV regressions can be used to consistently estimate the ATE, TOT, and TUT effects, along with each of the ingredients needed to construct the ASL and ASB. 
These results are used below to provide a recipe for cluster-robust inference for the ASG, ASL, and ASB effects.
%The first result provides a regression-based approach to estimate the ATE and TOT.

\begin{prop} 
\label{prop:TOTreg}
Under Assumption \ref{assump:randchoice}, $Y = \mathbbm{E}(Y_0) + \text{ATE}\times Z_1 + \text{TOT} \times Z_2 D + U$ where $\mathbb{E}(U|Z) = 0$.
Therefore, under standard regularity conditions, an IV regression of $Y$ on an intercept, $Z_1$ and $Z_2 D$ with instruments $(1, Z_0, Z_1)$ consistently estimates the ATE and TOT. 
\end{prop}

\begin{proof}
By Assumption \ref{assump:randchoice} (iii),
\begin{align*}
Y &= Z_0 Y_0 + Z_1 Y_1 + Z_2[(1 - C) Y_0 + CY_1] = (Z_0 + Z_2)Y_0 + Z_1 Y_1 + Z_2C(Y_1 - Y_0)\\
&= (Z_0 + Z_1 + Z_2)Y_0 + Z_1 (Y_1 - Y_0) + Z_2D(Y_1 - Y_0) = Y_0 + Z_1 (Y_1 - Y_0) + Z_2D(Y_1 - Y_0).
\end{align*}
since $Z_2 D = Z_2 C$ and $(Z_0 + Z_1 + Z_2) = 1$.
Thus, defining 
\[
U \equiv [Y_0 - \mathbbm{E}(Y_0)] + Z_1[(Y_1 - Y_0) - \text{ATE}] + Z_2 D[(Y_1 - Y_0) - \text{TOT}]
\]
by construction we have $Y = \mathbbm{E}(Y_0) + \text{ATE} \times Z_1 + \text{TOT} \times Z_2 D+ U$.
Now, since $Z_2 D = Z_2 C$ and $Z$ is independent of $(Y_1, Y_0)$ by Assumption \ref{assump:randchoice} (i), we have
\begin{align*}
\mathbbm{E}(U|Z) &= [\mathbbm{E}(Y_0|Z) - \mathbbm{E}(Y_0)]  + Z_1[\mathbbm{E}(Y_1 - Y_0|Z) - \text{ATE}] +  \mathbbm{E}\left[Z_2 D\left\{(Y_1 - Y_0) - \text{TOT}\right\}|Z\right]\\
&= [\mathbbm{E}(Y_0) - \mathbbm{E}(Y_0)]  + Z_1[\mathbbm{E}(Y_1 - Y_0) - \text{ATE}] + Z_2 \mathbbm{E}\left[C\left\{(Y_1 - Y_0) - \text{TOT}\right\}|Z\right]\\
&= Z_2 \mathbbm{E}\left[C\left\{(Y_1 - Y_0) - \text{TOT}\right\}|Z\right].
\end{align*}
Finally, by iterated expectations
\begin{align*}
\mathbbm{E}\left[C\left\{(Y_1 - Y_0) - \text{TOT}\right\}|Z\right] &=  \mathbbm{P}(C=1|Z) \left[\mathbbm{E}(Y_1 - Y_0|C=1,Z)  - \text{TOT}\right]\\
&= \mathbbm{P}(C=1)\left[\mathbbm{E}(Y_1 - Y_0|C=1) - \text{TOT} \right]=0
\end{align*}
since $Z$ is conditionally independent of $(Y_0, Y_1)$ given $C$, an implication of \ref{assump:randchoice} (i).
\end{proof}

%The next proposition provides regression-based estimates of the ATE and TUT.
\begin{prop}
\label{prop:TUTreg}
Under Assumption \ref{assump:randchoice} $Y = \mathbbm{E}(Y_1) + \text{ATE}\times -Z_0 + \text{TUT} \times -Z_2 (1 - D) + V$ where $\mathbb{E}(V|Z) = 0$.
Therefore, under standard regularity conditions, an IV regression of $Y$ on an intercept, $-Z_0$ and $-Z_2(1-D)$ with instruments $(1, Z_0, Z_1)$ provides consistent estimates of the ATE and TUT effects.
\end{prop}

\begin{proof}
By Assumption \ref{assump:randchoice} (iii),
\begin{align*}
Y &= Z_0 Y_0 + Z_1 Y_1 + Z_2[(1 - C) Y_0 + CY_1] = Z_0 Y_0 + Z_1 Y_1 + Z_2[(1 - C) (Y_0 - Y_1) + Y_1]\\
&= Z_0 Y_0 + (Z_1 + Z_2) Y_1 + Z_2(1 - C) (Y_0 - Y_1) = Z_0 Y_0 + (1 - Z_0) Y_1 + Z_2(1 - D) (Y_0 - Y_1) \\
&= Y_1 - Z_0 (Y_1 - Y_0) - Z_2(1 - D) (Y_1 - Y_0)
\end{align*}
since $Z_2(1 - C)= Z_2(1 - D)$ and $(Z_1 + Z_2) = 1 - Z_0$.
Thus, defining 
\[
V \equiv [Y_1 - \mathbb{E}(Y_1)] - Z_0[(Y_1 - Y_0) - \text{ATE}] - Z_2(1 - D)[(Y_1 - Y_0) - \text{TUT}]
\]
by construction we have $Y = \mathbb{E}(Y_1) + \text{ATE} \times -Z_0 + \text{TUT} \times -Z_2(1 - D) + V$.
Now, since $Z_2(1 - D) = Z_2 (1 - C)$ and $Z$ is independent of $(Y_0, Y_1)$ by Assumption \ref{assump:randchoice} (i),
\begin{align*}
\mathbb{E}(V|Z) &= [\mathbb{E}(Y_1|Z) - \mathbb{E}(Y_1)] - Z_0[\mathbb{E}(Y_1 - Y_0|Z) - \text{ATE}] - \mathbb{E}[Z_2(1 - D)\left\{(Y_1 - Y_0) - \text{TUT}\right\}|Z]\\
&= [\mathbb{E}(Y_1) - \mathbb{E}(Y_1)] - Z_0[\mathbb{E}(Y_1 - Y_0) - \text{ATE}] - Z_2\mathbb{E}[(1 - C)\left\{(Y_1 - Y_0) - \text{TUT}\right\}|Z]\\
&= -Z_2\mathbb{E}[(1 - C)\left\{(Y_1 - Y_0) - \text{TUT}\right\}|Z].
\end{align*}
Finally, by iterated expectations,
\begin{align*}
\mathbb{E}[(1 - C)\left\{(Y_1 - Y_0) - \text{TUT}\right\}|Z]&= \mathbb{P}(C=0|Z)\left[ \mathbb{E}(Y_1 - Y_0|C=0, Z) - \text{TUT}\right]\\
&= \mathbb{P}(C=0|Z) \left[ \mathbb{E}(Y_1 - Y_0|C=0) - \text{TUT}\right] = 0
\end{align*}
since $Z$ is conditionally independent of $(Y_0, Y_1)$ given $C$, an implication of Assumption \ref{assump:randchoice} (i).
\end{proof}

Since $\text{ASG} = \text{TOT} - \text{TUT}$, the preceding two propositions provide consistent estimates of $\text{ASG}$ effect.
The $\text{ASB}$ effect, $\mathbbm{E}(Y_0|C=1) - \mathbb{E}(Y_0|C=0)$, can likewise be estimated by taking the difference of coefficients across two linear IV regressions. %as shown in the following proposition.

\begin{prop}
\label{prop:ASBreg}
Under Assumption \ref{assump:randchoice}
\begin{align}
\label{eq:ASB0}
    (1 - D) Y &= \mathbbm{E}(Y_{0}) \times Z_0 + \mathbbm{E}(Y_{0}|C=0) \times (1 - D)Z_2 + U_{0}\\
\label{eq:ASB1}
    (1 - D) Y &= \mathbbm{E}(Y_{0}) \times(Z_0 + Z_2) + \mathbbm{E}(Y_{0}|C=1)\times -DZ_2 + U_{1}
\end{align}
where $\mathbbm{E}(U_0|Z) = \mathbbm{E}(U_1|Z) = 0$.
Thus, under standard regularity conditions, an IV regression of $(1 - D)Y$ on $Z_0$ and $(1 - D)Z_2$ with instruments $(Z_0, Z_2)$ and no intercept consistently estimates $\mathbb{E}(Y_0)$ and $\mathbb{E}(Y_0|C=0)$. 
Similarly, an IV regression of $(1 - D)Y$ on $(Z_0 + Z_2)$ and $-DZ_2$ with instruments $(Z_0, Z_2)$ and no intercept consistently estimates $\mathbb{E}(Y_0)$ and $\mathbb{E}(Y_0|C=1)$.
\end{prop}

\begin{proof}
Assumption \ref{assump:randchoice} (ii) implies $(1 - D) = Z_0 + Z_2(1 - C)$. Hence, by Assumption \ref{assump:randchoice} (iii),
\begin{align*}
(1 - D)Y &=  [Z_0 + Z_2 (1 - C)]\left\{Z_0 Y_0 + Z_1 Y_1 + Z_2[(1 - C) Y_0 + CY_1]\right\}\\
&= Z_0 Y_0 + Z_2 (1 - C) [(1 - C) Y_0 + C Y_1] = Z_0 Y_0 + Z_2 (1 - C) Y_0.
\end{align*}
because $Z_j^2 = Z_j$ for any $j$ and $Z_j Z_k = 0$ for any $j \neq k$ and, similarly, $(1 - C)^2 = (1 - C)$ and $C (1 - C) = 0$.
Therefore, since $Z_2 (1 - C) = Z_2 (1 - D)$,
\[
(1 - D)Y = Z_0 Y_0 + Z_2 (1 - D) Y_0, \quad
(1 - D)Y = (Z_0 + Z_2) Y_0 + (-DZ_2) Y_0. 
\]
Now, defining
\begin{align*}
U_0 &\equiv Z_0 [Y_0 - \mathbbm{E}(Y_0)] + Z_2(1 - D)[Y_0 - \mathbbm{E}(Y_0|C=0)]\\
U_1 &\equiv (Z_0 + Z_2)[Y_0 - \mathbbm{E}(Y_0)] + (-Z_2 D)[Y_0 - \mathbbm{E}(Y_0|C=1)]
\end{align*}
by construction, we have
\begin{align*}
Y &= \mathbbm{E}(Y_0) \times Z_0 + \mathbbm{E}(Y_0|C=0) \times Z_2(1 - D) + U_0\\
Y &= \mathbbm{E}(Y_0) \times (Z_0 + Z_2)+ \mathbbm{E}(Y_0|C=1) (-Z_2D) + U_1.
\end{align*}
Finally, since $Z_2(1-D) = Z_2(1 - C)$, and $Z$ is independent of $Y_0$,  
\begin{align*}
\mathbbm{E}(U_0|Z) &= Z_0[\mathbbm{E}(Y_0|Z) - \mathbbm{E}(Y_0)] + Z_2\mathbbm{E}\{(1 - C) [Y_0 - \mathbbm{E}(Y_0|C=0) |Z\}\\
&= Z_2 \mathbbm{E}[Y_0 - \mathbbm{E}(Y_0|C=0) | C = 0, Z] = 0
\end{align*}
where the final two steps follow by iterated expectations and the fact that  $Z$ is conditionally independent of $Y_0$ given $C$. 
Since $Z_2 D = Z_2 C$, a nearly identical argument gives
\begin{align*}
\mathbbm{E}(U_1|Z) &= (Z_0 + Z_2)[Y_0 - \mathbbm{E}(Y_0|Z)] -Z_2\mathbbm{E}\{C [Y_0 - \mathbbm{E}(Y_0|C=1) |Z\}\\
&= -Z_2 \mathbbm{E}[Y_0 - \mathbbm{E}(Y_0|C=0) | C = 1, Z] = 0. \qedhere
\end{align*}
\end{proof}


The next result shows that the $\text{ASL}$ effect, $\mathbbm{E}(Y_1|C=1) - \mathbb{E}(Y_1|C=0)$, can be estimated as the difference of coefficients across two linear IV regressions.

\begin{prop}
\label{prop:ASLreg}
Under Assumption \ref{assump:randchoice},
\begin{align}
\label{eq:ASL0}
    D Y &= \mathbbm{E}(Y_{1}) \times(Z_1 + Z_2) + \mathbbm{E}(Y_{1}|C=0)\times (D - 1)Z_2 + V_{0}\\
\label{eq:ASL1}
    D Y &= \mathbbm{E}(Y_{1}) \times Z_1 + \mathbbm{E}(Y_{1}|C=1)\times D Z_2+ V_{1}
\end{align}
where $\mathbbm{E}(V_0|Z) = \mathbbm{E}(V_1|Z) = 0$.
Thus, under standard regularity conditions, an IV regression of $DY$ on $(Z_1 + Z_2)$ and $-Z_2(1 - D)$ with instruments $(Z_1, Z_2)$ and no intercept provides consistent estimates of $\mathbb{E}(Y_1)$ and $\mathbb{E}(Y_1|C=1)$.
Similarly, an IV regression of $DY$ on $Z_1$ and $DZ_2$ with instruments $(Z_1, Z_2)$ and no intercept provides consistent estimates of $\mathbb{E}(Y_1)$ and $\mathbb{E}(Y_1|C=1)$. 
\end{prop}


\begin{proof}
By Assumption \ref{assump:randchoice}, $D = Z_1 + Z_2 C$. Hence, by Assumption \ref{assump:randchoice} (iii),
\begin{align*}
DY &=  (Z_1 + Z_2 C)\left\{Z_0 Y_0 + Z_1 Y_1 + Z_2[(1 - C) Y_0 + CY_1]\right\}\\
&= Z_1 Y_1 + Z_2 C[(1 -C) Y_0 + C Y_1] = Z_1 Y_1 + Z_2 C Y_1
\end{align*}
because $Z_j^2 = Z_j$ for any $j$ and $Z_j Z_k = 0$ for any $j \neq k$ and, similarly, $(1 - C)^2 = (1 - C)$ and $C (1 - C) = 0$. Therefore, since $Z_2 (1 - C) = Z_2 (1 - D)$,
\[
 DY = (Z_1 + Z_2)Y_1 + Z_2 (D - 1) Y_1, \quad
 DY = Z_1 Y_1 + Z_2 D Y_1.
\]
Now, defining
\begin{align*}
V_0 &= (Z_1 + Z_2)[Y_1 - \mathbbm{E}(Y_1)] + Z_2(D - 1)[Y_1 - \mathbbm{E}(Y_1|C=0)]\\
V_1 &= Z_1[Y_1 - \mathbbm{E}(Y_1)] + Z_2D[Y_1 - \mathbbm{E}(Y_1|C=1)]
\end{align*}
by construction we have
\begin{align*}
Y &= \mathbbm{E}(Y_1) \times (Z_1 + Z_2) + \mathbbm{E}(Y_1 | C = 0) \times Z_2 (D - 1) + V_0\\
Y &= \mathbbm{E}(Y_1) \times Z_1 + \mathbbm{E}(Y_1 | C = 1) \times Z_2 D + V _1.
\end{align*}
Finally, since $Z_2(1 - D) = Z_2(1 - C)$ and $Z$ is independent of $Y_1$,
\begin{align*}
\mathbbm{E}(V_0|Z) &= (Z_1 + Z_2) [\mathbbm{E}(Y_1|Z) - \mathbbm{E}(Y_1) ]  - Z_2 \mathbbm{E}\{(1 - C)  [Y_1 - \mathbbm{E}(Y_1 | C=0)|Z\} \\
&= -Z_2 \mathbbm{E}[Y_1 - \mathbbm{E}(Y_1 | C=0)|C=0, Z] = 0
\end{align*}
where the final two steps follow by iterated expectations and the fact that  $Z$ is conditionally independent of $Y_1$ given $C$. 
Since $Z_2 D = Z_2 C$, a nearly identical argument gives
\begin{align*}
\mathbbm{E}(V_1|Z) &= Z_1 [\mathbbm{E}(Y_1 | Z) - \mathbbm{E}(Y_1)]  + Z_2 \mathbbm{E}\{C  [Y_1 - \mathbbm{E}(Y_1 | C=1)|Z\} \\
&= Z_2 \mathbbm{E}[Y_1 - \mathbbm{E}(Y_1 | C=1)|C=1, Z] = 0. \qedhere
\end{align*}
\end{proof}




\subsection{Inference for ASG, ASB, and ASL}
\label{subsec:inference}
We now explain how to carry out cluster-robust inference for the ASG, ASB, and ASL effects, as implemented in our companion STATA package.
Each of these effects can be expressed as a difference of coefficients from two just-identified linear IV regressions.
The ASG effect is the difference of the \text{TOT} effect from Proposition \ref{prop:TOTreg} and the \text{TUT} effect from Proposition \ref{prop:TUTreg}.
Similarly, the ASB effect is the difference of $\mathbb{E}(Y_0|C=1)$ and $\mathbbm{E}(Y_0|C=0)$ from Proposition \ref{prop:ASBreg} while the ASL effect is the difference of $\mathbbm{E}(Y_1|C=1)$ and $\mathbbm{E}(Y_1|C=0)$ from Proposition \ref{prop:ASLreg}.
Within each pair of IV regressions the outcome variable and instrument set is identical; only the regressors differ. 
Since our estimators of all three effects share the same structure, our discussion of inference abstracts from the specific regressors and instruments used in each case to avoid duplication. 
To implement these results in practice without relying on our STATA package, simply substitute supply the regressors and instruments specified in the relevant propositions. 

Let $g = 1, ..., G$ index clusters and $i = 1, ..., N_g$ index individuals within a particular cluster $g$. 
In our experiment, a cluster is a branch-day combination and the experimentally-assigned treatment (control, forced, or choice arm) is assigned at the cluster level.
We assume that observations are iid across clusters but potentially correlated within cluster.
Now consider a pair of just-identified linear IV regressions given by
\[
Y_{ig} = \boldsymbol{X}_{1,ig}' \boldsymbol{\theta}_0 + U_{ig}, \quad
Y_{ig} = \boldsymbol{X}_{0,ig}' \boldsymbol{\theta}_1 + V_{ig}
\]
with common instrument vector $\boldsymbol{W}_{ig}$. 
Stacking observations in the usual manner, e.g.\
\[
\mathbf{W}_g' \equiv \begin{bmatrix}
\boldsymbol{W}_{1g} & \cdots & 
\boldsymbol{W}_{N_gg} 
\end{bmatrix}, \quad
\mathbf{W}' = \begin{bmatrix}
\mathbf{W}_1' & \cdots & \mathbf{W}_G'
\end{bmatrix},
\]
we can write the preceding equations in matrix form as
\[
\mathbf{Y} = \mathbf{X}_1\boldsymbol{\theta_1} + \mathbf{U}, \quad
\mathbf{Y} = \mathbf{X}_0\boldsymbol{\theta_0} + \mathbf{V}
\]
with instrument matrix $\mathbf{W}$.
Now, the IV estimators for $\boldsymbol{\theta}_1$ and $\boldsymbol{\theta}_0$ can be expressed as 
\begin{align*}
\widehat{\boldsymbol{\theta}}_1 
&= \left(\mathbf{W}'\mathbf{X}_1\right)^{-1}\mathbf{W}'\mathbf{Y}  = \boldsymbol{\theta}_1 + \left(\mathbf{W}'\mathbf{X}_1\right)^{-1}\mathbf{W}'\mathbf{U}\\ 
\widehat{\boldsymbol{\theta}}_0 
&= \left(\mathbf{W}'\mathbf{X}_0\right)^{-1}\mathbf{W}'\mathbf{Y} = \boldsymbol{\theta}_0 + \left(\mathbf{W}'\mathbf{X}_0\right)^{-1}\mathbf{W}'\mathbf{V}.
\end{align*}
Our parameter of interest is an element of the difference $(\theta_1 - \theta_0)$, so  it suffices to calculate the asymptotic distribution of $(\widehat{\boldsymbol{\theta}}_1 - \widehat{\boldsymbol{\theta}}_0)$.
Re-arranging the preceding two equations, we have
\begin{align*}
\sqrt{G} \left(\widehat{\boldsymbol{\theta}}_1  - \boldsymbol{\theta}_1\right) &= 
 \left(\frac{\mathbf{W}'\mathbf{X}_1}{G}\right)^{-1}\left(\frac{\mathbf{W}'\mathbf{U}}{\sqrt{G}}\right) = \left( \frac{1}{G} \sum_{g=1}^G \mathbf{W}_g' \mathbf{X}_{1,g}\right)^{-1}\left(\frac{1}{\sqrt{G}} \sum_{g=1}^G\mathbf{W}_g' \mathbf{U}_g\right) \\ 
\sqrt{G} \left(\widehat{\boldsymbol{\theta}}_0  - \boldsymbol{\theta}_0\right) &= 
 \left(\frac{\mathbf{W}'\mathbf{X}_0}{G}\right)^{-1}\left(\frac{\mathbf{W}'\mathbf{V}}{\sqrt{G}}\right) = \left( \frac{1}{G} \sum_{g=1}^G \mathbf{W}_g' \mathbf{X}_{0,g}\right)^{-1}\left(\frac{1}{\sqrt{G}} \sum_{g=1}^G\mathbf{W}_g' \mathbf{V}_g\right).
\end{align*}
Now, define $\mathbf{Q}_0 \equiv \mathbb{E}[\mathbf{W}_g'\mathbf{X}_{0g}]$ and $\mathbf{Q}_1 \equiv \mathbb{E}[\mathbf{W}_g'\mathbf{X}_{1g}]$ and suppose that these expectations exist and that the matrices $\mathbf{Q}_0$ and $\mathbf{Q}_1$ are invertible. Then, as $G \rightarrow \infty$
\[
 \left( \frac{1}{G} \sum_{g=1}^G \mathbf{W}_g' \mathbf{X}_{0,g}\right)^{-1} \rightarrow_p \mathbf{Q}_0^{-1}, \quad
 \left( \frac{1}{G} \sum_{g=1}^G \mathbf{W}_g' \mathbf{X}_{1,g}\right)^{-1} \rightarrow_p \mathbf{Q}_1^{-1} \quad
\]
by the continuous mapping theorem.
Now, by our experimental design and exclusion restriction, $\mathbf{W}_{ig}$ is independent of $U_{ig}$ both unconditionally and conditional on cluster size. 
Therefore,
\begin{align*}
\mathbb{E}[\mathbf{W}_g' \mathbf{U}_g] &=  \mathbb{E}\left[ \sum_{i=1}^{N_g} \mathbf{W}_{ig} U_{ig} \right] = \mathbb{E}_{N_g}\left[ \mathbb{E}\left\{\left. \sum_{i=1}^{N_g} \mathbf{W}_{ig} U_{ig} \right| N_g \right\}\right]\\
&= \mathbb{E}_{N_g}\left[ \sum_{i=1}^{N_g} \mathbb{E}(\mathbf{W}_{ig} U_{ig}|N_g)\right] = \mathbb{E}_{N_g}\left[ \sum_{i=1}^{N_g} \mathbb{E}(\mathbf{W}_{ig}U_{ig})\right] = 0
\end{align*}
by iterated expectations, and similarly smilarly, $\mathbb{E}[\mathbf{W}_g'\mathbf{V}_g] = \mathbf{0}$. 
Thus, under mild regularity conditions (e.g.\ finite fourth moments), as $G \rightarrow \infty$  we have
\[
\frac{1}{\sqrt{G}} \sum_{g=1}^G \mathbf{W}_g' \otimes \begin{bmatrix} \mathbf{U}_g \\ \mathbf{V}_g \end{bmatrix} \rightarrow_d \text{N}(\mathbf{0}, \boldsymbol{\Omega})
\]
where the heteroskedasticity-- and cluster--robust variance-covariance matrix $\boldsymbol{\Omega}$ is defined as
\[
\boldsymbol{\Omega} \equiv 
 \mathbb{E} \left[\left(\mathbf{W}_g \mathbf{W}_g' \right)\otimes \begin{pmatrix}
\mathbf{U}_g \mathbf{U}_g' & \mathbf{U}_g \mathbf{V}_g' \\
\mathbf{V}_g \mathbf{U}_g' & \mathbf{V}_g \mathbf{V}_g'  
\end{pmatrix} \right] \equiv \begin{bmatrix}
\Omega_{UU} & \Omega_{UV} \\
\Omega_{VU} & \Omega_{VV}
\end{bmatrix}.
\]
Therefore, the joint limiting distribution of $\widehat{\boldsymbol{\theta}}_1$ and $\widehat{\boldsymbol{\theta}}_0$ is given by
\[
\begin{bmatrix}
\sqrt{G}(\widehat{\boldsymbol{\theta}}_1 - \boldsymbol{\theta}_1)\\
\sqrt{G}(\widehat{\boldsymbol{\theta}}_0 - \boldsymbol{\theta}_0)
\end{bmatrix} \rightarrow_d
\begin{bmatrix}
\mathbf{Q}_1^{-1} & \mathbf{0} \\
\mathbf{0} & \mathbf{Q}_0^{-1} 
\end{bmatrix}
\begin{bmatrix}
\boldsymbol{\xi}_U \\ \boldsymbol{\xi}_V 
\end{bmatrix}, \quad
\begin{bmatrix}
\boldsymbol{\xi}_U \\ \boldsymbol{\xi}_V 
\end{bmatrix} \sim \text{N}\left(\begin{bmatrix} \mathbf{0} \\ \mathbf{0}\end{bmatrix},
\begin{bmatrix}
\Omega_{UU} & \Omega_{UV} \\
\Omega_{VU} & \Omega_{VV}
\end{bmatrix}\right)
\]
from which it follows by the continuous mapping theorem that
\[
\sqrt{G}\left[ (\widehat{\boldsymbol{\theta}}_1 - \widehat{\boldsymbol{\theta}}_0 ) - (\boldsymbol{\theta}_1 - \boldsymbol{\theta}_0)\right] 
\rightarrow_d
\begin{bmatrix}
\mathbf{Q}_1^{-1} & -\mathbf{Q}_0^{-1}
\end{bmatrix}
\begin{bmatrix}
\boldsymbol{\xi}_U \\ \boldsymbol{\xi}_V 
\end{bmatrix}.
\]
To use this result in practice, we estimate the standard errors of  $(\widehat{\boldsymbol{\theta}}_1 - \widehat{\boldsymbol{\theta}}_0)$ as the square root of the diagonal elements of the matrix
\[
\widehat{\text{Avar}}(\widehat{\boldsymbol{\theta}}_1 - \widehat{\boldsymbol{\theta}}_0) 
= \begin{bmatrix}
\left(\mathbf{W}'\mathbf{X}_1\right)^{-1} &
-\left(\mathbf{W}'\mathbf{X}_0\right)^{-1} 
\end{bmatrix}
\begin{bmatrix}
\mathbf{S}_{UU}& \mathbf{S}_{UV}\\
\mathbf{S}_{VU}& \mathbf{S}_{VV}
\end{bmatrix} 
\begin{bmatrix}
\left(\mathbf{X}_1'\mathbf{W}\right)^{-1} \\
-\left(\mathbf{X}_0'\mathbf{W}\right)^{-1} 
\end{bmatrix}
\]
where
\[
\mathbf{S}_{UU} \equiv \sum_{g=1}^G \mathbf{W}_g' \widehat{\mathbf{U}}_g \widehat{\mathbf{U}}_g' \mathbf{W}_g, \quad
\mathbf{S}_{UV} \equiv \sum_{g=1}^G \mathbf{W}_g' \widehat{\mathbf{U}}_g \widehat{\mathbf{V}}_g' \mathbf{W}_g, \quad
\mathbf{S}_{VU} \equiv \mathbf{S}_{UV}', \quad
\mathbf{S}_{VV} \equiv \sum_{g=1}^G \mathbf{W}_g' \widehat{\mathbf{V}}_g \widehat{\mathbf{V}}_g' \mathbf{W}_g
\]
and we define the IV residuals
\[
\widehat{\mathbf{U}}_g \equiv \mathbf{Y}_g - \mathbf{X}_{1,g}\widehat{\boldsymbol{\theta}}_1, \quad
\widehat{\mathbf{V}}_g \equiv \mathbf{Y}_g - \mathbf{X}_{0,g}\widehat{\boldsymbol{\theta}}_0.
\]
In our application the number of clusters, $G$, is large.
If desired, an \emph{ad hoc} degrees of freedom correction can be applied by multiplying the standard errors by $\sqrt{G/(G-1)}$.


\section{Testing the Exclusion Restriction}
\label{append:exclusion}


This section describes the testable implications of our exclusion restrictions: \eqref{eq:exclusion0} and \eqref{eq:exclusion1} from Section \ref{sec:potentialOutcomes}. 
To consider possible violations of these conditions, it is helpful to introduce some additional notation.
As above, let $Y_0 \equiv Y(d=0,z=0)$ and $Y_1 \equiv Y(d=1,z=1)$ denote the potential outcomes under \emph{forced treatment}: $Y_0$ is the potential outcome when forced into the status quo contract and $Y_1$ when forced into the commitment contract. 
Further let $Y_{0,2} \equiv Y(d=0,z=2)$ and $Y_{1,2} \equiv Y(d=1,z=2)$ denote the potential outcomes under \emph{free choice of treatment}: $Y_{0,2}$ is the potential outcome when choosing the status quo contract and $Y_{1,2}$ when choosing the commitment contract. 
Using this notation, \eqref{eq:exclusion0} becomes $Y_0 = Y_{0,2}$ and while \eqref{eq:exclusion1} becomes $Y_1 = Y_{1,2}$.
Without imposing these, part (iii) of Assumption \ref{assump:randchoice} becomes 
\[
Y = \mathbbm{1}(Z =0) Y_{0} + \mathbbm{1}(Z = 1)  Y_{1}  + \mathbbm{1}(Z = 2) \left[(1 - C) Y_{0,2} + C Y_{1,2} \right]
\]
but parts (i) and (ii) continue to hold.
Accordingly, parts (i)--(iii) of Lemma \ref{lem_randchoice} are unchanged, while parts (iv) and (v) become
\[
\mathbbm{E}(Y|D=0,Z=2) = \mathbbm{E}(Y_{0,2}|C=0), \quad
\mathbbm{E}(Y|D=1,Z=2) = \mathbbm{E}(Y_{0,1}|C=1).
\]
Using these expressions, the testable restrictions we consider here are as follows:
\begin{align}
\label{eq:exclusion_append0}
\mathbbm{E}(Y_0|C=0) &= \mathbbm{E}(Y_{0,2}|C=0) \\
\label{eq:exclusion_append1}
\mathbbm{E}(Y_1|C=1) &= \mathbbm{E}(Y_{1,2}|C=1).
\end{align}
Equation \ref{eq:exclusion_append0} is a restriction on the average potential outcomes of non-choosers that we use to point identify the TUT effect.
It says that someone who \emph{would} choose the control condition when given a choice, experiences the same potential outcome, on average, when \emph{assigned} to the control condition.
Equation \ref{eq:exclusion_append1} is a restriction on the average potential outcomes of choosers that we use to point identify the TOT effect.
It says that someone who \emph{would} choose the commitment contract when given a choice, experiences the same potential outcome, on average, when \emph{assigned} to this condition.
Because they refer to different groups of people, either of \eqref{eq:exclusion_append0}  and \eqref{eq:exclusion_append1} could hold when the other is violated.
For this reason we consider each in turn.
Our approach is closely related to arguments from \cite{huber_mellace} and \cite{BinaryRegressor}, among others.
%As such, the following is a heuristic explanation rather than a fully-rigorous proof.
%See the aforementioned papers for more details of how to make this argument fully rigorous.

Consider first \eqref{eq:exclusion_append0}.
Let $p \equiv \mathbbm{P}(C=1) = \mathbbm{P}(D=1|Z=2)$ denote the share of choosers in the population.
This value is point identified regardless of whether the exclusion restriction holds.
Because $Z$ was randomly assigned, a fraction $p$ of borrowers with $Z=0$ are choosers while the remaining $(1-p)$ are non-choosers.
It follows that, regardless of whether the exclusion restriction holds, the observed distribution of $Y|Z=0$ is a mixture of $Y_0|C=0$ and $Y_0|C=1$ with mixing weights $(1-p)$ and $p$.
This allows us to construct a pair of bounds for $\mathbb{E}(Y_0|C=0)$ as follows.
The non-choosers must lie \emph{somewhere} in the distribution of $Y|Z=0$.
Consider the two most extreme possibilities: they could occupy the bottom $(1-p)\times 100\%$ of the distribution or the top $(1-p)\times 100\%$ of the distribution.
For this reason, computing the average of the \emph{truncated} distribution of $Y|Z=0$, cutting out the top $p\times 100\%$, provides a lower bound for the average of $Y_0$ among non-choosers.
Similarly, cutting out the bottom $p \times 100\%$ provides an upper bound.
Let $y^0_{1-p}$ denote the $(1-p)$ quantile of $Y|Z=0$ and $y^0_{p}$ denote the $p$ quantile of the same distribution.
Using this notation, the bounds are given by
\[
\mathbb{E}\left(Y|Z=0, Y\leq y^0_{1-p}\right)\leq \mathbb{E}(Y_0|C=0) \leq \mathbb{E}\left(Y|Z=0, Y \geq y^0_p\right) 
\]
These bounds do not rely on the exclusion restriction.
Under Equation \ref{eq:exclusion_append0}, however, we know that $\mathbb{E}(Y_0|C=0)=\mathbb{E}(Y|D=0,Z=2)$.
Therefore, if the exclusion restriction for non-choosers holds, we must have
\begin{equation}
\mathbb{E}\left(Y|Z=0, Y\leq y^0_{1-p}\right)\leq \mathbb{E}(Y|D=0,Z=2) \leq \mathbb{E}\left(Y|Z=0, Y \geq y^0_p\right).
\label{eq:testable0}
\end{equation}
Equation \ref{eq:testable0} provides a pair of testable implications of \eqref{eq:exclusion_append0}.
If either inequality is violated, then the exclusion restriction for non-choosers fails.
In our experiment, $\widehat{p} = \widehat{\mathbbm{P}}(D=1|Z=2) = 0.11$. For the APR outcome we estimate
\[
\widehat{\mathbbm{E}}(Y_\text{APR}|Z=0,Y_\text{APR}\leq y^0_{0.89}) = 0.48, \quad
\widehat{\mathbbm{E}}(Y_\text{APR}|Z=0, Y_\text{APR}\geq y^0_{0.11}) = 0.62.
\]
Since $\widehat{\mathbbm{E}}(Y_\text{APR}|D=0,Z=2) = 0.58$ falls between these bounds, we find no evidence against the exclusion restriction for non-choosers. 
Repeating this exercise for the financial cost outcome, we see that $\widehat{\mathbbm{E}}(Y_\text{FC}|D=0,Z=2) = 973$ likewise falls between 
\[
\widehat{\mathbbm{E}}(Y_\text{FC}|Z=0,Y_\text{FC}\leq y^0_{0.89}) = 626, \quad
\widehat{\mathbbm{E}}(Y_\text{FC}|Z=0, Y_\text{FC}\geq y^0_{0.11}) = 1046 
\]
so we again find no evidence against the exclusion restriction for non-choosers.

We can use an analogous approach to construct testable implications for \ref{eq:exclusion_append1}. 
Because $Z$ was randomly assigned, a fraction $p$ of participants with $Z = 1$ are choosers so the distribution of $Y|Z=1$ is a mixture of $Y_1|C=1$ and $Y_1|C=0$ with mixing weights $p$ and $1 -p$.
The participants with $C = 1$ must lie \emph{somewhere} in the distribution of $Y_0|Z=1$ so again consider the two most extreme cases: they could occupy the bottom or the top $p\times 100\%$ of the distribution. 
Hence,
\[
\mathbbm{E}\left(Y|Z=1,Y\leq y^1_{p}\right) \leq \mathbbm{E}(Y_1|C=1) \leq \mathbbm{E}\left(Y|Z=1,Y \geq y^1_{1-p}\right)
\]
where $y^1_p$ and $y^1_{1-p}$ are the $p$ and $1 - p$ quantiles of the distribution of $Y|Z=1$.
As above, these bounds do not rely on the exclusion restriction.
If Equation \ref{eq:exclusion_append1} holds, however, we know that $\mathbb{E}(Y|D=1,Z=2) = \mathbb{E}(Y_1|C=1)$, yielding the following testable implications for choosers
\begin{equation}
\mathbbm{E}\left(Y|Z=1,Y\leq y^1_{p}\right) \leq \mathbbm{E}(Y|D=1,Z=2) \leq \mathbbm{E}\left(Y|Z=1,Y \geq y^1_{1-p}\right).
\label{eq:testable1}
\end{equation}
If either inequality is violated, then the exclusion restriction from Equation \ref{eq:exclusion_append1} fails.
Again, in our experiment $\widehat{p}=0.11$. For the APR outcome we estimate
\[
\widehat{\mathbbm{E}}(Y|Z=1,Y\leq y^1_{0.11}) = 0.06, \quad
\widehat{\mathbbm{E}}(Y|Z=1,Y\geq y^1_{0.89}) = 1.28
\]
Since $\widehat{\mathbbm{E}}(Y_\text{APR}|D=1,Z=2) = 0.43$ falls between these bounds, we find no evidence against the exclusion restriction for the choosers. 
Repeating this exercise for the financial cost outcome, we see  that $\widehat{\mathbbm{E}}(Y_\text{FC}|D=1,Z=2)= 570$ likewise falls between
\[
\widehat{\mathbbm{E}}(Y|Z=1,Y\leq y^1_{0.11}) = 53, \quad
\widehat{\mathbbm{E}}(Y|Z=1,Y\geq y^1_{0.89}) = 2934
\]
so we again find no evidence against the exclusion restriction for the choosers.

The aforementioned bounds for $\mathbbm{E}(Y_0|C=0)$ and $\mathbbm{E}(Y_1|C=1)$ can alternatively be used construct partial identification bounds for the TOT and TUT effects that do not rely on our exclusion restrictions. If the TOT or TUT expression from \ref{append:randchoice} fall \emph{outside} the partial identification bounds, this is equivalent to finding a violation of \ref{eq:testable0} or \ref{eq:testable1}. We implement these bounds for the TOT and TUT in our companion STATA package.

\end{document}