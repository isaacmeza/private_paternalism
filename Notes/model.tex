\documentclass[11pt]{article}
\usepackage{amsmath,amssymb,amsthm,fullpage}
\usepackage{bm}
\usepackage{mathtools}
\usepackage{color}
\usepackage{algorithm}
\usepackage{algpseudocode}

\newtheorem{theorem}{Theorem}
\newtheorem{lemma}{Lemma}
\newtheorem{proposition}{Proposition}
\newtheorem{remark}{Remark}

\begin{document}

\section{Model Setup}

\paragraph{Periods and Loan.}
Time is discrete, \(t \in \{1,2,3\}\). In period \(t=1\), the borrower receives a loan of size \(L > 0\). She has exogenous incomes \(\{y_1,y_2,y_3\}\) in each period. We allow for a \emph{prepayment} of size \(k \in [0, L]\) in period~2, and a final repayment decision in period~3. Two interest rates apply:
\[
R_2 > 0 
\quad (\text{applies to the period-2 prepayment}), \]
\[R_3 > 0 
\quad (\text{applies to the period-3 repayment of remaining balance}).
\]
Hence, if the borrower prepays \(k\) in period~2, she pays \(R_2 \cdot k\). If there remains \(L - k\) principal going into period~3, repaying it fully costs \(R_3 \cdot (L - k)\).

\paragraph{Preferences.}
We assume \emph{quasi-hyperbolic} discounting with parameters \((\beta, \delta)\), where \(0 < \delta < 1\) and \(0 < \beta \le 1\). The instantaneous utility is given by some \(u(\cdot)\). Thus, from the perspective of period~1, the total utility over consumption \(\{c_1, c_2, c_3\}\) plus an extra payoff \(v\) if she repays by \(t=3\) is
\begin{align}
U 
\;=\;
u(c_1) 
\;+\;
\beta\,\delta \, u(c_2)
\;+\;
\beta\,\delta^2 
\Bigl(
   u(c_3) 
   \;+\; 
   \mathbf{1}_{\{\text{repay at }t=3\}} \cdot v
\Bigr),
\tag{1}
\end{align}
where \(\mathbf{1}_{\{\text{repay at }t=3\}}\) indicates \(1\) if the borrower fully repays in period~3, and \(0\) otherwise.

\paragraph{Behavioral Types.}
\begin{itemize}
\item \emph{Rational}: \(\beta = 1\).  
\item \emph{Present-Biased Sophisticate}: \(0 < \beta < 1\) and \emph{foresees} that future selves also discount by \(\beta\).
\item \emph{Present-Biased Na\"{i}f}: \(0 < \beta < 1\) but \emph{believes (incorrectly)} that future selves use \(\beta=1\).
\end{itemize}
Moreover, the borrower may be \emph{overconfident} about future income, believing \((y_2,y_3)\sim \mathbb{P}'\) that is more optimistic than the true distribution \(\mathbb{P}\). A \emph{realistic} borrower uses the correct \(\mathbb{P}\).

\paragraph{Commitment vs.\ No Commitment.}
At period \(1\), the borrower chooses
\[
C 
\;\in\; 
\{\text{Commit},\, \text{No-Commit}\}.
\]
Under \textbf{No-Commit}, the borrower \emph{may} prepay \(k\) in \(t=2\), but faces no penalty for skipping it. Under \textbf{Commit}, if she skips \(k\) in \(t=2\), she must pay a fee \(\phi \ge 0\) in period~3 \emph{if} she ultimately repays.  

\section{Period-by-Period Decisions}

\subsection{Period 3: Repay or Default}

Let the \emph{outstanding principal} be \(L-k\) if the borrower prepays \(k\) in period~2, or \(L\) if she does not. If she is in the commitment regime but skips \(k\), she also owes a fee \(\phi\) when repaying. Thus:

\[
\text{Consumption if Repay} 
\;=\; 
y_3 \;-\; R_3 \times (\text{outstanding}) 
\;-\; 
\phi \cdot \mathbf{1}_{\{\text{commit} \,\wedge\, \text{skipped }k\}},
\]
plus the continuation payoff \(v\). If she \textbf{defaults}, consumption is \(y_3\) with no \(v\). Hence she repays if and only if
\[
u\Bigl(
  y_3 - R_3 \cdot (\text{outstanding})
  \;-\; 
  \phi \cdot \mathbf{1}_{\{\text{commit}\,\wedge\,\text{skipped }k\}}
\Bigr)
\;+\; v
\quad>\quad
u(y_3).
\]

\subsection{Period 2: Pay $k$ or Not}

In period~2, the borrower’s discount factor toward period~3 is \(\beta\,\delta\).  Denote
\[
\text{Pay-$k$ Utility} 
\;=\; 
u\bigl(y_2 - R_2\,k\bigr)
\;+\;
\beta\,\delta\,
\mathbb{E}\Bigl[
  \max\{\,u(y_3 - R_3(L - k)) + v,\;u(y_3)\}
\Bigr],
\]
\[
\text{Not-Pay Utility (No-Commit)} 
\;=\;
u(y_2)
\;+\;
\beta\,\delta\,
\mathbb{E}\Bigl[
  \max\{\,u(y_3 - R_3\,L) + v,\;u(y_3)\}
\Bigr].
\]
Under \textbf{No-Commit}, she prepays \(k\) if and only if
\[
\text{Pay-$k$ Utility} 
\;>\;
\text{Not-Pay Utility (No-Commit)}.
\]
Under \textbf{Commit}, skipping \(k\) adds \(\phi\) if she repays at \(t=3\). Thus
\[
\text{Skip-$k$ Utility (Commit)}
\;=\;
u(y_2)
\;+\;
\beta\,\delta\,
\mathbb{E}\Bigl[
  \max\bigl\{
    u(y_3 - R_3\,L - \phi) + v,\;
    u(y_3)
  \bigr\}
\Bigr].
\]
So in \textbf{Commit}, she pays \(k\) iff
\[
\text{Pay-$k$ Utility} 
\;>\;
\text{Skip-$k$ Utility (Commit)}.
\]

\subsection{Period 1: Choose Commitment or Not}

At \(t=1\), the borrower compares the ex-ante \emph{perceived} expected utilities of \textbf{No-Commit} vs.\ \textbf{Commit}:
\[
U_{\text{NC}}
\;=\;
u(y_1)
\;+\;
\beta_{\text{hat}}\,\delta 
\;\mathbb{E}_{(y_2,y_3)}\Bigl[
   \max\{\text{Pay-$k$ Utility},\;\text{Not-Pay Utility (No-Commit)}\}
\Bigr],
\]
\[
U_{\text{C}}
\;=\;
u(y_1)
\;+\;
\beta_{\text{hat}}\,\delta 
\;\mathbb{E}_{(y_2,y_3)}\Bigl[
  \max\{\text{Pay-$k$ Utility},\;\text{Skip-$k$ Utility (Commit)}\}
\Bigr].
\]
Here, \(\beta_{\text{hat}}=1\) if the borrower is \emph{na\"{\i}ve} (incorrectly assuming future \(\beta=1\)), whereas \(\beta_{\text{hat}}=\beta\) if she is \emph{sophisticated} or rational (\(\beta=1\)). Also, \(\mathbb{E}_{(y_2,y_3)}\) might be taken under the true distribution \(\mathbb{P}\) (if realistic) or a biased \(\mathbb{P}'\) (if overconfident). She chooses \(\text{Commit}\) iff \(U_{\text{C}} > U_{\text{NC}}\).

% \section{Thresholds in Each Period}

% \paragraph{Period 3 (Repay vs.\ Default).} Define the function
% \[
% D_3(y_3,\text{outstanding},\phi_{\text{commit}}) 
% \;=\;
% u\bigl(y_3 - R_3\,(\text{outstanding}) - \phi_{\text{commit}}\bigr)
% \;+\; v 
% \;-\; 
% u(y_3).
% \]
% She repays iff \(D_3(\cdot) > 0\). Otherwise she defaults.

% \paragraph{Period 2 (Pay-$k$ vs.\ Not Pay).} 
% In either regime, define
% \[
% D_2^{\text{(pay)}}(y_2)
% \;=\;
% u(y_2 - R_2\,k)
% \;+\;
% \beta\,\delta \,
% \mathbb{E}\bigl[\text{repay-or-default payoff if outstanding }(L-k)\bigr]\]
% \[\qquad \;-\;
% \Bigl(
%   u(y_2)
%   \;+\;
%   \beta\,\delta \,
%   \mathbb{E}\bigl[\text{repay-or-default payoff if outstanding }L \bigr]
% \Bigr).
% \]
% Under \textbf{No-Commit}, she pays \(k\) iff \(D_2^{\text{(pay)}}(y_2) > 0\).  
% Under \textbf{Commit}, replace the “repay payoff” for the \((L\text{-skipped},\phi)\) case accordingly. She pays \(k\) iff
% \[
% D_2^{\text{(pay)}}(y_2) 
% \;>\; 
% D_2^{\text{(skip, commit)}}(y_2),
% \]
% where \(D_2^{\text{(skip, commit)}}\) includes the \(\phi\) penalty if repaying.

% \paragraph{Period 1 (Commit vs.\ No-Commit).}
% Define
% \[
% D_1 
% \;=\;
% U_{\text{C}} - U_{\text{NC}}
% \;\;=\;\;
% \bigl[u(y_1) + \beta_{\text{hat}}\delta \,\mathbb{E}(\dots)\bigr] 
% \;-\;
% \bigl[u(y_1) + \beta_{\text{hat}}\delta \,\mathbb{E}(\dots)\bigr].
% \]
% She chooses \(\text{Commit}\) iff \(D_1>0\).

\section{Outcomes of Interest}

Once the borrower’s type (rational, naive, or sophisticated; realistic or overconfident) is specified, along with the realized \((y_2,y_3)\), the above thresholds determine each period’s decision. We then define:

\begin{enumerate}
\item \textbf{Probability of Default}:
\[
\Pr[\text{default}] \;=\; \mathbb{E}\bigl[\mathbf{1}_{\{\text{default at }t=3\}}\bigr].
\]
\item \textbf{Average Payment}:
\[
\mathbb{E}\bigl[m_2 + m_3\bigr],
\]
where \(m_2 \in \{0,\,kR_2\}\) is the actual period-2 payment (including interest factor \(R_2\)), and \(m_3 \in \{0,\,(\!L\!-\!k\!)\,R_3, LR_3\}\) is the period-3 repayment if it occurs.
\item \textbf{Probability of Choosing Commitment}:
\[
\Pr[C = \text{Commit}]
\]
\item \textbf{Financial Cost} \(F\):
\[
F \;=\;
\begin{cases}
\displaystyle
\phi \cdot \mathbf{1}_{\{C=\text{commit} \,\wedge\, m_2=0\}}
\;+\;v \;-\;L \;+\;(m_2 + m_3),
&
\text{if default occurs at }t=3,
\\[6pt]
\displaystyle
\phi \cdot \mathbf{1}_{\{C=\text{commit} \,\wedge\, m_2=0\}}
\;+\;\bigl[\,(m_2 - k) + (m_3 - (L-k))\bigr],
&
\text{if repaying in full at }t=3.
\end{cases}
\]
\noindent
Here, \(m_2 = R_2\,k\) if the borrower prepays, else \(0\); \(m_3 = R_3\,(L - k)\) if the borrower repays in \(t=3\) and $m_2 =  R_2\,k$, if $m_2=0$, then \(m_3 = R_3\,L \), otherwise it is \(m_3=0\). The term \(\phi\) arises only if the borrower is in the commitment regime, skipped \(k\), and repays. In the event of default, we add \(\,(v - L + m_2 + m_3)\). In the event of repayment, we effectively sum the “interest” portions \(\,(m_2 - k) + (m_3 - (L-k))\).
\end{enumerate}



\end{document}
