\usetheme[numbering=fraction, sectionpage=none,titleformat=smallcaps]{metropolis}


%\useoutertheme{smoothbars}
%\setbeamercolor{mini frame}{fg=white, bg=black}
%\setbeamercolor{section in head/foot}{fg=white, bg=black}

% !TEX TS-program = pdflatex
% !TEX encoding = UTF-8 Unicode


%%%%%%%%%%%%%%%%%%%%%%%%
% Usual LaTeX Packages %
%%%%%%%%%%%%%%%%%%%%%%%%

\usepackage{amsmath}
\usepackage{amsfonts}
\usepackage{amssymb}
\usepackage{graphicx}
\usepackage{mathrsfs} 			% For Weinberg-esque letters
\usepackage{cancel}				% For "SUSY-breaking" symbol
\usepackage{slashed}            % for slashed characters in math mode
\usepackage{bbm}                % for \mathbbm{1} (unit matrix)
\usepackage{amsthm}				% For theorem environment
\usepackage{multirow}			% For multi row cells in table
\usepackage{arydshln} 			% For dashed lines in arrays and tables
\usepackage{multirow}
\usepackage{multicol}
\usepackage{caption}
\usepackage{subcaption}
\usepackage{bigstrut}
\usepackage{setspace}
\usepackage{endnotes}
\usepackage{etex}
\usepackage{lmodern}
\usepackage{booktabs}
\usepackage{graphics}
\usepackage[flushleft]{threeparttable}
%\usepackage{enumitem}
%\setlist[itemize]{itemsep=2mm}
\usepackage{array}
\usepackage{color}
\usepackage{colortbl}
\usepackage{hyperref}
\usepackage{ifplatform}
\usepackage{dcolumn}

%% FONT
\usepackage[default,osfigures,scale=0.95]{opensans} %% Alternatively
%% use the option 'defaultsans' instead of 'default' to replace the
%% sans serif font only.

% TO SHOW NOTES ON ANOTHER SCREEN
%\usepackage{pgfpages}
%\setbeameroption{hide notes}
%\setbeameroption{show notes on second screen=right}
%FONT PACKAGE ==============================================
\usepackage[utf8]{inputenc}
\usepackage[T1]{fontenc}
\usepackage{appendixnumberbeamer}

% Doing Better Math
\usefonttheme{professionalfonts} % required for mathspec



% HYPERREF =========================================================
% HyperRef Set up
\hypersetup{colorlinks=true,
           linkcolor=brown,
           bookmarks=true,
           anchorcolor=green,
           menucolor=cyan,
           citecolor=.,
           urlcolor=cyan,
           final=true
           }


% Code for Hiding Columns in Tables
\newcolumntype{H}{>{\setbox0=\hbox\bgroup}c<{\egroup}@{}}


\graphicspath{{images/}}	% Put all images in this directory. Avoids clutter.



% SOME COMMANDS THAT I FIND HANDY
% \renewcommand{\tilde}{\widetilde} % dinky tildes look silly, doesn't work with fontspec
\newcommand{\comment}[1]{\textcolor{comment}{\footnotesize{#1}\normalsize}} % comment mild
\newcommand{\Comment}[1]{\textcolor{Comment}{\footnotesize{#1}\normalsize}} % comment bold
\newcommand{\COMMENT}[1]{\textcolor{COMMENT}{\footnotesize{#1}\normalsize}} % comment crazy bold
\newcommand{\Alert}[1]{\textcolor{Alert}{#1}} % louder alert
\newcommand{\ALERT}[1]{\textcolor{ALERT}{#1}} % loudest alert
\newcommand{\red}{\textcolor[rgb]{1,0,0}}
\renewcommand\appendixname{Appendix}
\setbeamerfont{frametitle}{size=\large}


% BIBLIOGRAPHY (OPTION: BIBLATEX) ===========================================
\usepackage{csquotes}
\usepackage[natbib = true, backend = biber, style  = authoryear-icomp]{biblatex}
\addbibresource{References2.bib}
%\addbibresource{References.bib}
\addbibresource{library.bib}

\renewcommand{\cite}{\citet}
\ifmacosx
% UNCOMMENT IF COMPILING ON A MAC (No Fix in Linux but to install
% Biblatex 3.6)
% Problem with Commas in Biblatex
% http://tinyurl.com/yclhtz6s
\DeclareDelimFormat[cbx@textcite]{nameyeardelim}{\addspace}
% Redefine \cite to be \citet so that the silly bracket problem does
% not arise when using biblatex.
\fi


%%% HUGE KLUDGE JUST TO MAKE THE COMMA DISAPPEAR WHEN CITING PAPERS
%%% (BIBLATEX PROBLEM ONLY). COMMENT OUT WHEN COMPILING ON THE MAC
%%% WHICH DOESNT SEEM TO NEED THIS.
% \iflinux

% \makeatletter 
% \renewbibmacro*{textcite}{% 
%   \iffieldequals{namehash}{\cbx@lasthash} 
%     {\iffieldundef{shorthand} 
%        {\ifthenelse{\iffieldequals{labelyear}{\cbx@lastyear}\AND 
%                     \(\value{multicitecount}=0\OR\iffieldundef{postnote}\)} 
%           {\setunit{\addcomma}% 
%            \usebibmacro{cite:extrayear}} 
%           {\setunit{\compcitedelim}% 
%            \usebibmacro{cite:labelyear+extrayear}% 
%            \savefield{labelyear}{\cbx@lastyear}}} 
%        {\setunit{\compcitedelim}% 
%         \usebibmacro{cite:shorthand}% 
%         \global\undef\cbx@lastyear}} 
%     {\ifnameundef{labelname} 
%        {\iffieldundef{shorthand} 
%           {\usebibmacro{cite:label}% 
%            \setunit{% 
%              \global\booltrue{cbx:parens}% 
%              \addspace\bibopenparen}% 
%            \ifnumequal{\value{citecount}}{1} 
%              {\usebibmacro{prenote}} 
%              {}% 
%            \usebibmacro{cite:labelyear+extrayear}} 
%           {\usebibmacro{cite:shorthand}}} 
%        {\printnames{labelname}% 
%         \setunit{% 
%           \global\booltrue{cbx:parens}% 
%           \addspace\bibopenparen}% 
%         \ifnumequal{\value{citecount}}{1} 
%           {\usebibmacro{prenote}} 
%           {}% 
%         \iffieldundef{shorthand} 
%           {\iffieldundef{labelyear} 
%              {\usebibmacro{cite:label}} 
%              {\usebibmacro{cite:labelyear+extrayear}}% 
%            \savefield{labelyear}{\cbx@lastyear}} 
%           {\usebibmacro{cite:shorthand}% 
%            \global\undef\cbx@lastyear}}% 
%      \stepcounter{textcitecount}% 
%      \savefield{namehash}{\cbx@lasthash}}% 
%   \setunit{% 
%     \ifbool{cbx:parens} 
%       {\bibcloseparen\global\boolfalse{cbx:parens}} 
%       {}% 
%     \textcitedelim}} 
% \makeatother
% \fi
% ;



% BIBLIOGRAPHY (OPTION: BIBTEX) ===========================================
%\usepackage[elide]{natbib}
%\usepackage{bibentry}

% TITLE PAGE =============================================
\setbeamertemplate{title page}[default]
\setbeamercolor{titlelike}{parent=frametitle,fg=crimsonred}
\setbeamercolor{author}{fg=black,bg=white}
\setbeamercolor{institute}{fg=black,bg=white}
\setbeamercolor{date}{fg=black,bg=white}


% FRAME TITLE ==============================================
\setbeamercolor{background canvas}{bg=white}
\setbeamercolor{normal text}{fg=black}
\setbeamercolor{frametitle}{bg=white,fg=crimsonred}
\setbeamerfont*{frametitle}{series=\bfseries,size=\Large}

% BULLETPOINTS =========================================
\setbeamertemplate{itemize item}{\small\raise.2ex\hbox{\donotcoloroutermaths$-$}}
\setbeamertemplate{itemize subitem}{\small\raise.1ex\hbox{\donotcoloroutermaths$\circ$}}
\setbeamertemplate{itemize subsubitem}{\scriptsize\raise.1ex\hbox{\donotcoloroutermaths$\bullet$}}
\setbeamercolor{itemize item}{fg=lightash}
\setbeamercolor{itemize subitem}{fg=ash}
\setbeamercolor{itemize subsubitem}{fg=ash}
\setbeamercolor{enumerate item}{fg=ash}
\setbeamertemplate{enumerate item}{(\arabic{enumi})}
\setbeamercolor{enumerate subitem}{fg=ash}

% TABLE OF CONTENTS ========================================
\setbeamertemplate{section in toc}{\hypersetup{linkcolor=.}\Huge{\alert{\textbf{\colorbox{crimsonred}{\textcolor{white}{\inserttocsectionnumber}}\enskip\textcolor{crimsonred}{\inserttocsection}}}}}
\setbeamertemplate{section in toc shaded}{\hypersetup{linkcolor=.}\colorbox{lightgrey}{\textcolor{white}{\inserttocsectionnumber}}\enskip\inserttocsection}
\setbeamercolor{section in toc shaded}{fg=lightgrey}
%\setbeamertemplate{section in toc shaded}[default][50]

% NAVIGATION BAR ===============================================
\setbeamertemplate{headline}{%
\hypersetup{linkcolor=white}
\begin{beamercolorbox}[colsep=1.5pt]{upper separation line head}
\end{beamercolorbox}
\begin{beamercolorbox}{section in head/foot}
    \vskip2pt\insertsectionnavigationhorizontal{\paperwidth}{\hskip0pt plus1fill}{\hskip0pt plus1fill}\vskip2pt
\end{beamercolorbox}%
%\begin{beamercolorbox}[ht=10pt]{subsection in head/foot}%
%    \vskip2pt\insertsubsectionnavigationhorizontal{\paperwidth}{}{\hskip0pt plus1filll}\vskip2pt
%\end{beamercolorbox}%
\begin{beamercolorbox}[colsep=1.5pt]{lower separation line head}
\end{beamercolorbox}
}
\setbeamercolor{section in head/foot}{fg=white, bg=black}
\setbeamercolor{section in head/foot}{bg=ash}




% COLOR DEFINITIONS ==============================================
\definecolor{bbva}{RGB}{0,76,147}		% Neurtal red, good for dark or light bg
\definecolor{crimsonred}{RGB}{153,0,0}		% Neurtal red, good for dark or light bg
\definecolor{darkcrimsonred}{RGB}{105,0,0}	
\definecolor{darkcharcoal}{RGB}{25,25,25}		% Darker gray
\definecolor{charcoal}{RGB}{51,51,51}		% Darker gray
\definecolor{ash}{RGB}{100,100,100}			% medium gray
\definecolor{paleblue}{RGB}{0,102,102}		% More of an `ocean' color
\definecolor{turtlegreen}{RGB}{230,124,0}	% A more neutral green
\definecolor{paleale}{RGB}{204,204,102}		% Only for dark BG
\definecolor{lager}{RGB}{140,110,10}		% Use instead of pale ale for white BG
\definecolor{regal}{RGB}{90,0,120}			% A more neutral purple
\definecolor{jeans}{RGB}{20,30,150}			% A more neutral blue
\definecolor{red}{RGB}{137,37,46}
\definecolor{charcoal}{RGB}{82,89,85}
\definecolor{ash}{RGB}{100,100,100}
\definecolor{lightash}{RGB}{140,140,140}
\definecolor{gold}{RGB}{160,129,51}
\definecolor{navy}{RGB}{15,62,86}
\definecolor{crimsonred}{RGB}{153,0,0}
\definecolor{lightgrey}{RGB}{218,218,218}

\definecolor{azu_timeline}{rgb}{.267,  .329,  .416}
\definecolor{ama_timeline}{rgb}{1,  .753,  0}
\definecolor{nar_timeline}{rgb}{ .776,  .349,  .067}
\definecolor{roj_timeline}{RGB}{189, 0, 13}
\definecolor{ver_timeline}{rgb}{.329,  .51,  .208}







%% FOOTLINE ====
\setbeamertemplate{footline}{%
  \begin{beamercolorbox}[sep=0.5em,wd=\paperwidth,leftskip=0.5em,rightskip=0.5em]{footlinecolor}
    %\includegraphics[scale=0.07,height=10pt]{images/image1.png}
  $-$ Frequent Payments \hfill%
    \tiny{\insertframenumber/\inserttotalframenumber}
    %\includegraphics[scale=0.25,height=10pt]{images/image2.png}
  \end{beamercolorbox}%
}
\setbeamercolor{footlinecolor}{fg=white,bg=ash}






%% Does a Progress Bar for the Talk at the Bottom of the Page
%% http://tex.stackexchange.com/questions/59742/progress-bar-for-latex-beamer

% \setbeamercolor{progress bar progress}{use=progress bar,bg=progress bar.fg}
% \defbeamertemplate{footline}{progress bar}{
%   \dimen0=\paperwidth
%   \multiply\dimen0 by \insertframenumber
%   \divide\dimen0 by \inserttotalframenumber
%   \edef\progressbarwidth{\the\dimen0}

%   \leavevmode%
%   \begin{beamercolorbox}[wd=\paperwidth,ht=1.75ex,dp=1ex]{progress
%       bar}
%     \begin{beamercolorbox}[wd=\progressbarwidth,ht=1.75ex,dp=1ex]{progress bar progress}
%     \end{beamercolorbox}%
%     \insertframenumber{} / \inserttotalframenumber
%   \end{beamercolorbox}%
% }
% \setbeamertemplate{footline}[progress bar]
% \setbeamercolor{progress bar}{fg=blue!50!black,bg=white!50!black}
% ;
%%%%%%%%%%%%%%%

%\usetikzlibrary{backgrounds}
%\usetikzlibrary{mindmap,trees}	% For mind map
% http://www.texample.net/tikz/examples/computer-science-mindmap/

%%% Local Variables:
%%% mode: latex
%%% TeX-master: t
%%% End:


% %%%% MAURICIO ADDED THESE
% \setbeamerfont{frametitle}{size=\large}
% \setbeamertemplate{navigation symbols}{}
% \newcolumntype{L}[1]{>{\raggedright\let\newline\\\arraybackslash\hspace{0pt}}m{#1}}
% \newcolumntype{C}[1]{>{\centering\let\newline\\\arraybackslash\hspace{0pt}}m{#1}}
% \newcolumntype{R}[1]{>{\raggedleft\let\newline\\\arraybackslash\hspace{0pt}}m{#1}}

% \def\zapcolorreset{\let\reset@color\relax\ignorespaces}
% \def\colorrows#1{\noalign{\aftergroup\zapcolorreset#1}\ignorespaces}

% \newcolumntype{H}{>{\setbox0=\hbox\bgroup}c<{\egroup}@{}}


% %center text in tables
% \newcommand{\specialCellCenter}[2][c]{\begin{tabular}[#1]{@{}c@{}}#2\end{tabular}}
% \newcommand{\specialcell}[2][l]{\begin{tabular}[#1]{@{}l@{}}#2\end{tabular}}
% \def\sym#1{\ifmmode^{#1}\else\(^{#1}\)\fi}
% \setbeamertemplate{note page}[compress]
% \setbeamerfont{note page}{size=\tiny}
% \setbeameroption{hide notes}
% \setbeamersize{text margin left=0.2in,text margin right=0.2in}
% \setbeamertemplate{footline}{}


%%%%% HACER HIGHLIGHT DE DIFERNETES PARTES

\usepackage[beamer,customcolors]{hf-tikz}
\usetikzlibrary{calc}
\tikzset{hl/.style={%
    set fill color=red!80!black!40,
    set border color=red!80!black,
  },
}
\RenewDocumentCommand{\tikzmarkin}{r<> o m D(){\belowrightoff} D(){\aboveleftoff}}{%
  \IfNoValueTF{#2}{%true-val
    \only<#1>{\tikz[remember picture,overlay]
      \draw[line width=1pt,rectangle,disable rounded corners,fill=\fcol,draw=\bcol]
      (pic cs:#3) ++(#4) rectangle (#5) node [anchor=base] (#3){}
      ;}%
  }{%false-val
    \only<#1>{\tikz[remember picture,overlay]
      \draw[line width=1pt,rectangle,disable rounded corners,fill=\fcol,draw=\bcol,#2]
      (pic cs:#3) ++(#4) rectangle (#5) node [anchor=base] (#3){}
      ;}}%
}




