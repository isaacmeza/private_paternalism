 \documentclass[11pt]{article}


\usepackage{soul}
\usepackage{natbib}
\usepackage{hyperref}
\usepackage{graphicx}             
\graphicspath{{./Figuras/}}

\usepackage{makecell}
\usepackage[margin=1.0in]{geometry}
\usepackage{float}                
\usepackage{amsmath}
\usepackage{amscd}
\usepackage{amsfonts}
\usepackage{amssymb}
\usepackage{bbm}
\usepackage{booktabs}
\usepackage{nameref}
\usepackage{multirow}
\usepackage[nokeyprefix]{refstyle}
\usepackage{rotating}
\usepackage{threeparttable}
\usepackage{lscape}
\usepackage{enumerate}
\usepackage{afterpage}
\usepackage{caption}
\usepackage{subcaption}
\usepackage{epstopdf}
\epstopdfDeclareGraphicsRule{.tiff}{png}{.png}{convert #1 \OutputFile}
\AppendGraphicsExtensions{.tiff}

\epstopdfDeclareGraphicsRule{.tif}{png}{.png}{convert #1 \OutputFile}
\AppendGraphicsExtensions{.tif}

\usepackage{tikz}
\usetikzlibrary{shapes.geometric, arrows}
\usetikzlibrary{calc}
\usetikzlibrary{matrix}

\tikzset{ 
    table/.style={
        matrix of nodes,
        row sep=-\pgflinewidth,
        column sep=-\pgflinewidth,
        nodes={
            rectangle,
            draw=black,
            align=center
        },
        minimum height=1.5em,
        text depth=0.5ex,
        text height=2ex,
        nodes in empty cells,
%%
        every even row/.style={
            nodes={fill=gray!20}
        },
        column 1/.style={
            nodes={text width=2em,font=\bfseries}
        },
        row 1/.style={
            nodes={
                fill=black,
                text=white,
                font=\bfseries
            }
        }
    }
}


\usepackage{colortbl}

\newtheorem{theorem}{Theorem}
\newtheorem{claim}[theorem]{Claim}



\usepackage{anyfontsize}
%%% HELPER CODE FOR DEALING WITH EXTERNAL REFERENCES
\usepackage{xr}
\makeatletter
\newcommand*{\addFileDependency}[1]{
  \typeout{(#1)}
  \@addtofilelist{#1}
  \IfFileExists{#1}{}{\typeout{No file #1.}}
}
\makeatother

\newcommand*{\myexternaldocument}[1]{
    \externaldocument{#1}
    \addFileDependency{#1.tex}
    \addFileDependency{#1.aux}
}

%\myexternaldocument{OA}

%%%%%%%%%%%%%%%%%%%%%%%%%%%%%%%% DOCUMENT
\begin{document}


\title{Frequent Payment \thanks{}}
\author{Joyce Sadka \and Enrique Seira \and Isaac Meza }
\date{This draft:  \today \\[2 cm]}

%\vspace{.5in}


\maketitle
\begin{abstract}
XXX 
\end{abstract}


We want to claim that for all increasing utility functions $u(\cdot)$:
\begin{align}
\label{fosd}
    \mathbb{E}_{sq}(u(FC))\geq \mathbb{E}_{fee}(u(FC))
\end{align}
as the FC is seen as a `bad'. It can be proven that relation (\ref{fosd}) is equivalent to 
\[F_{sq}(FC)\leq F_{fee}(FC)\]
where $F$ is the distribution function.

\begin{theorem}
The following are equivalent:
\begin{enumerate}[(a)]
    \item For every weakly decreasing utility function $u$: $E_{sq}(u(FC))\leq E_{fee}(u(FC))$
    \item $F_{sq}(FC)\leq F_{fee}(FC)$ 
\end{enumerate}
\end{theorem}
\begin{proof} \;

[$(a)\Longrightarrow (b)$] Suppose  $(b)$ does not hold. So there exists $FC^*$ such that $F_{sq}(FC^*)>F_{fee}(FC^*)$, Define $u:=\mathds{1}_{FC\leq FC^*}$. Then
\[E_{sq}(u(FC)) = \int u(FC)dF_{sq} = F_{sq}(FC^*)>F_{fee}(FC^*)= \int u(FC)dF_{fee} =E_{fee}(u(FC))\]
which contradicts $(a)$.\\

[$(a)\Longleftarrow (b)$] On the other hand for $u$ weakly decreasing,

\[\int u(y(FC))dF_{sq}(y(FC)) = \int u(y(FC))dF_{fee}(FC) \leq \int u(FC)dF_{fee}(FC)\]
with $y(FC) = F_{sq}^{-1}F_{fee}(FC)$.
\end{proof}

Note the proof was done in the case of absolutely continuous and strictly increasing distribution functions $F_{sq}$ and $F_{fee}$.



\begin{theorem}
Let $F$ and $G$ be the cumulative distributions of two alternative log-normal prospects. The following are equivalent:

\begin{enumerate}[(a)]
    \item For every weakly decreasing utility function $u$: $E_{F}(u(FC))\leq E_G(u(FC))$
    \item $E_F \log(FC)\geq E_G\log(FC)$ and $Var_F \log(FC)= Var_G\log(FC)$
\end{enumerate}

\end{theorem}

\begin{proof}
See Theorem 4 in \cite{lognormal_dominance}.
\end{proof}



Consider a two period model. The agent chooses to 

A linear model is sufficient to highlight the basic mechanism, and provides a reasonable approximation for small stakes


In the two period model, there is only one decision: whether the agent saves or not. 


\begin{prop}
Absent shocks in the first period, recovery is realized for sufficiently low probability of income shocks, i.e. when
\[\lambda\leq \frac{\beta b-p+(1-\beta)+D}{\beta(b-p)} =: \lambda_{max}\]
\end{prop}


\begin{cor}
\begin{enumerate}[(i)]
    \item $\frac{\partial \lambda_{max}}{\partial \beta}>0 $ : Probability of recovery in neoclassical individuals ($\beta = 1$) is higher than in present bias individuals.
    \item $\frac{\partial \lambda_{max}}{\partial D}>0 $ : In present bias individuals, the probability of recovery is higher in the Fee-Forcing ($D>0$) contract.  
\end{enumerate}
\end{cor}

\clearpage
\bibliographystyle{authordate1}
%\bibliographystyle{amsalpha}
%\bibliographystyle{AER}

\bibliography{References}
\end{document}